\documentclass[a4paper]{article}

\usepackage[utf8]{inputenc}
\usepackage[L7x]{fontenc}
\usepackage[lithuanian]{babel}
\usepackage{lmodern}
\usepackage{amsmath}
\usepackage{amssymb}
\usepackage{theorem}
\usepackage{bm}
\usepackage[unicode]{hyperref}

\begin{document}
\begin{titlepage}
\centerline{ \large VILNIAUS UNIVERSITETAS}
\bigskip
\centerline{\large MATEMATIKOS IR INFORMATIKOS FAKULTETAS}
\smallskip

\centerline{\large  EKONOMETRINĖS ANALIZĖS KATEDRA}
\vskip 200pt
\centerline{ \large Monika \textsc{Šeštokaitė} ir \large Simona \textsc{Plonytė}}
\vskip 50pt
\centerline{\bf \Large Kursinio darbo pavadinimas}
\bigskip
\vskip 50pt
\hfill Ekonometrija, III kursas, I grupė
\vskip 100pt
\centerline{\large VILNIUS 2011}
\end{titlepage}

\pagebreak

\centerline{\bf \Large Ekonometrinio kursinio darbo planas}
\section{ Įvadas - duomenų pristatymas}

*Be dealine'o \\

	1.1. Aprašomi turimi duomenys - iš kur paimti, koks periodas, ar yra visos reikšmės ir t.t. \\

	1.2. Kursinio darbo tikslas - sudaryti modelius, parinkti tinkamiausią, galbūt prognozuoti ir gauti kitas reikiamas išvadas

\section{ Turinys - ekonometrinė duomenų analizė} 
* iki spalio 5 dienos \\

	2.1. Turimų duomenų grafikai\\

	2.2. Duomenų ypatybių nustatymas - ar turi vienetinę šaknį, ar reikalinga transformacija, etc\\

	2.3. Randamos kintamųjų priklausomybės su įvairiais modeliais\\

	2.4. Modelių liekanų, informacinių kriterijų tikrinimas, geriausio modelio paieška\\

\section{Rezultatai}
*po spalio 5 d.\\

	3.1. Sudaryto ekonometrinio modelio interpretacija\\


	3.2. Galimas ekonometrinio modelio panaudojimas\\

	3.3. (galbūt prognozė)\\

\section{Išvados}

* 3 ir 4 skyriai iki lapkričio 10 d. \\

	4.1 kas padaryta ir kas nepadaryta, problemos.\\

\section{Literatūra}
\section{Priedai}
	

\end{document}