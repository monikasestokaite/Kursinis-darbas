\documentclass[a4paper]{article}

\usepackage[utf8]{inputenc}
\usepackage[L7x]{fontenc}
\usepackage[lithuanian]{babel}
\usepackage{lmodern}
\usepackage{amsmath}
\usepackage{amssymb}
\usepackage{theorem}
\usepackage{bm}
\usepackage[unicode]{hyperref}
\usepackage{color}

\begin{document}
\begin{titlepage}
\centerline{ \large VILNIAUS UNIVERSITETAS}
\bigskip
\centerline{\large MATEMATIKOS IR INFORMATIKOS FAKULTETAS}
\smallskip

\centerline{\large  EKONOMETRINĖS ANALIZĖS KATEDRA}
\vskip 200pt
\centerline{ \large Monika \textsc{Šeštokaitė} ir \large Simona \textsc{Plonytė}}
\vskip 50pt
\centerline{\bf \Large Kursinio darbo pavadinimas}
\bigskip
\vskip 50pt
\hfill Ekonometrija, III kursas, I grupė
\vskip 100pt
\centerline{\large VILNIUS 2011}
\end{titlepage}

\pagebreak

\centerline{\bf \Large Ekonometrinio projekto darbo planas}
\centerline{pagal laiką}

\section{ Iki spalio 4 d. -- duomenų analizė }


\begin{enumerate}
\item {Susirasti duomenis;}
\item {Aprašyti turimus duomenis - iš kur paimti, koks periodas, ar yra visos reikšmės ir t.t.;}
\item {Išbrėžti turimų duomenų grafikus;}
\item {Duomenų ypatybių nustatymas - ar turi vienetinę šaknį, ar reikalinga transformacija ir t.t.;}
\item {Kintamųjų priklausomybių nustatymas įvairiais modeliais;}
\item {Modelių liekanų, informacinių kriterijų tikrinimas;}
\item {Geriausio modelio radimas;}
\item {{\color{red} Prognozė (jeigu reikia)};}
\end{enumerate}


\section{ Iki lapkričio 3 d. -- pirminis variantas}
\subsection{lapkričio 3-10 d. pataisymai}

\begin{enumerate}
\item {Pasidaryti "griaučius" su skyrių pavadinimais;}
\item {Pradinis turinio variantas;}
\item {Suformuluoti kursinio darbo tikslą;} 
\textit{Turbūt užrašyti jį, nes susiformuluoti reikėtų darbo pradžioje :D}
\item {Modelio interpretacija ir rezultatų pateikimas;}
\item {Išvadų formulavimas - kas padaryta ir kas nepadaryta, problemos;}
\item {Įžanga;}
\end{enumerate}


\section{{\color{red} Iki lapkričio $\pm$ 24 d.} -- galutinis variantas}

\begin{enumerate}
\item {Mažiau svarbios informacijos iškėlimas į "Priedų" skyrelį;}
\item {Literatūros sąrašas;}
\item {Prezentacijos paruošimas, paskutiniai pataisymai;}

\end{enumerate}


\end{document}