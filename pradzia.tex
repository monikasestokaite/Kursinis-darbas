\documentclass[a4paper]{article}

\usepackage[utf8]{inputenc}
\usepackage[L7x]{fontenc}
\usepackage[lithuanian]{babel}
\usepackage{lmodern}
\usepackage{float}
\usepackage{setspace}
\usepackage{amsmath}
\usepackage{amssymb}
\usepackage{theorem}
\usepackage{bm}
\usepackage[unicode]{hyperref}
\usepackage{color}
\onehalfspacing
\begin{document}
\begin{titlepage}
\centerline{ \large VILNIAUS UNIVERSITETAS}
\bigskip
\centerline{\large MATEMATIKOS IR INFORMATIKOS FAKULTETAS}
\smallskip

\centerline{\large  EKONOMETRINĖS ANALIZĖS KATEDRA}
\vskip 200pt
\centerline{ \large Monika \textsc{Šeštokaitė} ir \large Simona \textsc{Plonytė}}
\vskip 50pt
\centerline{\bf \Large CAPM ir akcijų portfelio konstravimas}
\bigskip
\vskip 25pt
\centerline{Kursinio darbo vadovas prof. \large {Remigijus} \textsc{Leipus}}
\vskip 25pt
\hfill Ekonometrija, III kursas, I grupė
\vskip 100pt
\centerline{\large VILNIUS 2011}
\end{titlepage}

\pagebreak

\centerline{\bf \Large Ekonometrinio projekto pradinė duomenų analizė}

\section{ Tikslas}
Šioje analizėje bus išbrėžti duomenų grafikai, paskaičiuoti svarbiausi vidurkiai, dispersijos, standartiniai nuokrypiai ir koreliacijos. Po modelių sudarymo aukso, Microsoft ir Apple akcijų grąžoms bus įvertinti ir interpretuoti koeficientai, $ R^2 $, liekanos. Patikrinsime hipotezes apie $ \alpha $  ir $ \beta $ reikšmes. Taip pat sukonstruosime kelis akcinio kapitalo portfelius iš nagrinėjamų akcijų bei juos įvertinsime.

\section{ Duomenys}
Analizei pasirinkome aukso kainas bei Microsoft ir Apple korporacijų akcijų kainas. Taip pat buvo reikalingos Standard\&Poor 500 indekso reikšmės, bei JAV vyriausybės leidžiamų trumpalaikių (30 dienų) iždo vekselių grąžos. Iš kainų ir indekso reikšmių, paskaičiavome grąžas, kuriomis rėmėmės tolimesnėje analizėje. Duomenys -- dieniniai, su reikšmėmis nuo 2001 rugpjūčio 1d. iki 2011 liepos 29 d. Nedarbo dienų duomenys nebuvo pateikti, o trūkstamos reikšmės, kai visi likę tos dienos duomenys žinomi, užpildytos tiesiškai interpoliuojant.\\
\indentČia  Standard\&Poor 500 indekso reikšmių pokyčiai atitiks bendro finansų rinkos investicinio portfelio grąžas $ r_m $ , o JAV vyriausybės leidžiamų trumpalaikių (30 dienų) iždo vekselių grąžos --       (\textbf{ishversk risk-free rate of return}) $ r_f $. Taip pat dar bus naudojama visos rinkos rizikos premija -- $ (r_m - r_j) $ ir j-ojo vertybinio popieriaus rizikos premija -- $ (r_j - r_f) $, kur $ r_j $ -- vertybinio popieriaus $ j $ grąža.

\section{ CAPM }
CAPM \textit{(angl. Capital Asset Pricing Model)} -- tai modelis, apibūdinantis santykį tarp investicinės rizikos ir laukiamos akcinio kapitalo grąžos. \\
\indent Kadangi investuotojas nori ne tik maksimalizuoti akcinio kapitalo grąžas, tačiau ir minimizuoti riziką, tai rizika dažniausiai laikomas standartinis nuokrypis nuo laukiamos grąžos $ {\sigma}^2 $.\\
\indent Pagrindinė modelio priklausomybė yra tiesinė:
 $$ r_j - r_f =\alpha_j + \beta_j\cdot(r_m - r_f) + \epsilon_j, $$ 
kur $ \beta  =\frac{\sigma_{jm}}{\sigma^2_m} $, $ \sigma_{jm} $ -- kovariacija tarp akcijos $ j $ grąžų ir rinkos grąžų, o $ \sigma^2_m $ -- rinkos grąžų dispersija.\\
\indent Taikant CAPM modelį konstruojant akcinio kapitalo portfelį, svarbų vaidmenį vaidima diversifikavimas -- investavimas į kelias finansines priemones. Taip galima suskaidyti riziką ir optimalizuoti portfelį.


\end{document}
