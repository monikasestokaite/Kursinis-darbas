\documentclass[a4paper]{article}

\usepackage[utf8]{inputenc}
\usepackage[L7x]{fontenc}
\usepackage[lithuanian]{babel}
\usepackage{lmodern}
\usepackage{float}
\usepackage{setspace}
\usepackage{amsmath}
\usepackage{amssymb}
\usepackage{theorem}
\usepackage{bm}
\usepackage[unicode]{hyperref}
\usepackage{color}
\onehalfspacing
\begin{document}
\begin{titlepage}
\centerline{ \large VILNIAUS UNIVERSITETAS}
\bigskip
\centerline{\large MATEMATIKOS IR INFORMATIKOS FAKULTETAS}
\smallskip

\centerline{\large  EKONOMETRINĖS ANALIZĖS KATEDRA}
\vskip 200pt
\centerline{ \large Monika \textsc{Šeštokaitė} ir \large Simona \textsc{Plonytė}}
\vskip 50pt
\centerline{\bf \Large CAPM ir akcijų portfelio konstravimas}
\bigskip
\vskip 25pt
\centerline{Kursinio darbo vadovas prof. \large {Remigijus} \textsc{Leipus}}
\vskip 25pt
\hfill Ekonometrija, III kursas, I grupė
\vskip 100pt
\centerline{\large VILNIUS 2011}
\end{titlepage}

\pagebreak

\centerline{\bf \Large Ekonometrinio projekto pradinė duomenų analizė}

\section{ Tikslas}
Šioje analizėje bus išbrėžti duomenų grafikai, paskaičiuoti svarbiausi vidurkiai, dispersijos, standartiniai nuokrypiai ir koreliacijos. Po modelių sudarymo aukso, Microsoft ir Apple akcijų grąžoms bus įvertinti ir interpretuoti koeficientai, $ R^2 $, liekanos. Patikrinsime hipotezes apie $ \alpha $  ir $ \beta $ reikšmes. Taip pat sukonstruosime kelis akcinio kapitalo portfelius iš nagrinėjamų akcijų bei juos įvertinsime.

\section{ Duomenys}
Analizei pasirinkome aukso kainas bei Microsoft ir Apple korporacijų akcijų kainas. Taip pat buvo reikalingos Standard\&Poor 500 indekso reikšmės, bei JAV vyriausybės leidžiamų trumpalaikių (30 dienų) iždo vekselių grąžos. Iš kainų ir indekso reikšmių, paskaičiavome grąžas, kuriomis rėmėmės tolimesnėje analizėje. Duomenys -- dieniniai, su reikšmėmis nuo 2001 rugpjūčio 1d. iki 2011 liepos 29 d. Nedarbo dienų duomenys nebuvo pateikti, o trūkstamos reikšmės, kai visi likę tos dienos duomenys žinomi, užpildytos tiesiškai interpoliuojant.\\
\indentČia  Standard\&Poor 500 indekso reikšmių pokyčiai atitiks bendro finansų rinkos investicinio portfelio grąžas $ r_m $ , o JAV vyriausybės leidžiamų trumpalaikių (30 dienų) iždo vekselių grąžos --       (\textbf{ishversk risk-free rate of return}) $ r_f $. Taip pat dar bus naudojama visos rinkos rizikos premija -- $ (r_m - r_j) $ ir j-ojo vertybinio popieriaus rizikos premija -- $ (r_j - r_f) $, kur $ r_j $ -- vertybinio popieriaus $ j $ grąža.

\section{ CAPM }
CAPM \textit{(angl. Capital Asset Pricing Model)} -- tai modelis, apibūdinantis santykį tarp investicinės rizikos ir laukiamos akcinio kapitalo grąžos. \\
\indent Kadangi investuotojas nori ne tik maksimalizuoti akcinio kapitalo grąžas, tačiau ir minimizuoti riziką, tai rizika dažniausiai laikomas standartinis nuokrypis nuo laukiamos grąžos $ {\sigma}^2 $.\\
\indent Pagrindinė modelio priklausomybė yra tiesinė:
 $$ r_j - r_f =\alpha_j + \beta_j\cdot(r_m - r_f) + \epsilon_j, $$ 
kur $ \beta  =\frac{\sigma_{jm}}{\sigma^2_m} $, $ \sigma_{jm} $ -- kovariacija tarp akcijos $ j $ grąžų ir rinkos grąžų, o $ \sigma^2_m $ -- rinkos grąžų dispersija.\\
\indent Taikant CAPM modelį konstruojant akcinio kapitalo portfelį, svarbų vaidmenį vaidima diversifikavimas -- investavimas į kelias finansines priemones. Taip galima suskaidyti riziką ir optimalizuoti portfelį.


\end{document}

****************************************************

išvados po aktyvų modeliais:

Kapitalo aktyvų įvertinimo modelio 5 skyrelyje gal reikia prodyt kokius modelius sudarysim:

Microsoft rizikos premija $= \alpha + \beta \times $ rinkos rizikos premija

Apple rizikos premija $= \alpha + \beta \times $ rinkos rizikos premija

Aukso rizikos premija $= \alpha + \beta \times $ rinkos rizikos premija

Išvados: abiejų įmonių Microsoft Corp. ir Apple Inc. akcijų rizikos premijos panašiai priklauso nuo rinkos svyravimų - rinkos rizikos premijai pasikeitus 1\%, įmonių akcijų rizikos premija taip pat pakis apie 1\%. Kadangi Apple Inc. akcijų grąža didesnė, tai iš dviejų, panašius koeficientus turinčių įmonių, verta pasirinkti Apple Inc. Investavus į šios įmonės akcijas 100 \$, galima gauti 0,18 \$ per dieną. Tai pelningiausias aktyvas, bet kartu ir rizikingiausias.

Aukso rizikos premija rodo atvirkštinę nestiprią priklausomybę nuo rinkos rizikos premijos - taigi į auksą palanku investuoti krizės laikotarpiu. Pagal modelio pateiktą koeficientą, rinkos rizikos premijai nukritus 1\%, aukso rizikos premija išauga 0,02. Vis dėlto koeficiento $\beta$ p-reikšmė rodo, jog koeficientas nereikšmingas, tad galime sušvelninti sąlygą ir teigti, kad rinkos rizikos premijai nukritus 1\%, aukso rizikos premija nekinta arba priklauso nuo kitų faktorių. Į auksą investavus 100 \$, galima gauti 0,07 \$ per dieną.

Išvados apie portfelių modelius:

Labiausiai nuo rinkos svyravimų priklauso portfelis, sudarytas iš Microsoft Corp. ir Apple Inc. kompanijų akcijų. Į jį pelninga investuoti rinkos pakilimo laikotarpiu, nes rinkos rizikos premijai pakilus 1\%, portfelio rizikos premija taip pat išaugs 1\%. Investavus 100 \$, galima gauti 0,06\$ per dieną. Tačiau jis visiškai nepelningas krizės laikotarpiu. Tada investuotojui palankiau rinktis pirmąjį portfelį,sudarytą iš aukso ir Apple Inc. akcijų, kuris nuo rinkos svyravimų priklauso mažiausiai iš keturių portfelių. Rinkos rizikos premijai nukritus 1\%, šio portfelio rizikos premija pagal modelį nukris tik 0.17 \%, o į jį investavus 100 \$, galima gauti 0,09\$ per dieną - tai yra didžiausia grąža iš visų keturių portfelių.

Overall išvados:

1. Naudodami CAP modelį, patikrinome aukso bei dviejų kompanijų - Microsoft Corp. ir Apple Inc. akcijų rizikos premijų priklausomybę nuo rinkos rizikos premijų. Auksas nuo rinkos svyravimų priklauso mažiausiai (arba, kaip rodo modelis, iš vis nepriklauso), o kompanijų akcijos svyruoja panašiai kaip ir rinka.

2. Radome optimalias, t.y. mažiausią riziką turinčias, kombinacijas tarp dviejų aktyvų ir patikrinome jų rizikos premijos priklausomybę nuo rinkos rizikos premiją. Geriausias portfelis sudarytas iš aukso ir Apple Inc. akcijų. Jo pelningumas didžiausias ir jis mažiausiai priklauso nuo rinkos nestabilumo. Kita vertus, pagal CAP modelį, pelningas gali būti ir antras portfelis, su sąlyga, kad rinkos rizikos premija nuolat kyla.

3. Jei investuotojas gali rinktis ne tik investiciją į kurį nors rinkos aktyvą, bet ir į nerizikingą iždo vekselį, tai pagal CAP modelį ir kitus rodiklius (pelningumą, rizikos minimalumą) tai yra geriausia investicija. Trisdešimties dienų JAV iždo vekselis turi didžiausią grąžą, mažiausią riziką, mažai priklauso nuo rinko svyravimų, o bet kuri portfelio kombinacija su iždo vekseliu taip pat nerizikinga iri pelningesnė už kitų aktyvų portfelių kombinacijas. Tačiau verta pastebėti, kad tokie rezultatai atspindi duomenis nuo 2001 iki 2011 metų, o per pakilimo laikotarpį iždo vekselių pelningumas buvo itin didelis. Nuo 2008 metų pelningumas smarkiai krito, nors rizika nepakito ir liko maža.


