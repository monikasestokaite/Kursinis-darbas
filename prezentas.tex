%\documentclass[pdf,lt,slideColor,colorBG,noaccumulate,nototal,final]{prosper}
%% Norint matyti lietuviskas raides reikia susiinstaliuoti lietuvybes
%% arba unikodo palaikyma. Sitas dokumentas yra standartinėje ISO-8859-13 koduoteje. 
\documentclass[utf8,hyperref={unicode},xcolor=dvipsnames]{beamer}
%\documentclass[utf8]{beamer}

\mode<presentation>
{

\usecolortheme[named=Apricot]{structure} 
 
\usetheme[height=11mm]{Rochester} 
\setbeamertemplate{items}[ball] 
\setbeamertemplate{blocks}[rounded][shadow=true] 
\setbeamertemplate{navigation symbols}{} 
  % or whatever (possibly just delete it)
   \setbeamertemplate{footline}
   {%
     \leavevmode%
    \hbox{\begin{beamercolorbox}[wd=.5\paperwidth,ht=2.5ex,dp=1.125ex,leftskip=.3cm 	plus1fill,rightskip=.3cm]{author in head/foot}%
    \usebeamerfont{author in head/foot}\insertshortauthor
    \end{beamercolorbox}%
    \begin{beamercolorbox}[wd=.5\paperwidth,ht=2.5ex,dp=1.125ex,leftskip=.3cm,rightskip=.3cm plus1fil]{title in head/foot}%
    \usebeamerfont{title in head/foot}\insertshorttitle \hfill p.
	\insertpagenumber\enspace iš \insertdocumentendpage\enspace
    \end{beamercolorbox}}%
  \vskip0pt%
}
%\hfill\insertshorttitle\enspace -- p. \insertpagenumber\enspace iš
%\insertdocumentendpage\enspace%
%   }
    
}
 \setbeamertemplate{navigation symbols}{}



\usepackage[english]{babel}
%\usepackage[utf8x]{inputenc}
\usepackage[L7x]{fontenc}
\usepackage{lmodern}
\usepackage{amsmath}
\usepackage{amssymb}
%\usepackage{theorem}
\usepackage{bm}
\usepackage{graphicx}

\newcommand{\ab}[1]{#1_{\alpha}}
\newcommand{\wab}[2][\delta]{w_{\alpha}(#2,#1)}
\newcommand{\normab}[1]{\lVert#1\rVert_{\alpha}}
\newcommand{\eps}{\varepsilon}
\newcommand{\sprod}[1]{\langle #1 \rangle}
\DeclareMathOperator{\diam}{diam}
%\renewcommand{\theenumi}{\roman{enumi}}
%\renewcommand{\labelenumi}{\theenumi)}
\theoremstyle{change}\newtheorem{teorema}{Teiginys}
\theoremstyle{change}\newtheorem{salyga}{}
%	\vspace*{20pt}
%	\vspace*{20pt}
\DeclareMathOperator{\seq}{seq}
\DeclareMathOperator{\Var}{Var}
\DeclareMathOperator{\tr}{tr}
\newcommand{\ds}[1]{\displaystyle{#1}}
\newcommand{\dlt}[2]{\Delta^{(#1)}_{#2}}
\newcommand{\norms}[1]{\lVert#1\rVert_{\alpha}^{\seq}}
\newcommand{\normh}[1]{\lVert#1\rVert}
\newcommand{\norma}[1]{\lVert #1\rVert_{\alpha}}

\newcommand{\skirt}[2]{\Delta^{(#1)}_{#2}}
\newcommand{\cp}{\buildrel P\over\longrightarrow}
\renewcommand{\theenumi}{\roman{enumi}}
\newcommand{\R}{\mathbb{R}}
\newcommand{\E}{\mathbf{E}\,} % expectation operator
\newcommand{\bv}{\bm{v}}

\newcommand{\T}{T}
\newcommand{\n}{{\bm{n}}}
\newcommand{\jj}{{\bm{j}}}
\newcommand{\kk}{{\bm{k}}}
\newcommand{\bt}{\bm{t}}
\newcommand{\bu}{\bm{u}}
\newcommand{\B}{\bm{B}}
\newcommand{\N}{\mathbb{N}}
\newcommand{\bi}{\bm{i}}
\newcommand{\p}{\bm{\pi}}
\newcommand{\one}{{\bm{1}}}

\newcommand{\nni}{{\bm{n},\bm{i}}}
\newcommand{\nj}{{\bm{n},\bm{j}}}
\newcommand{\nk}{{\bm{n},\bm{k}}}
\newcommand{\kn}{\bm{k}_{\bm{n}}}

\newcommand{\vv}{\bm{\mathrm{v}}}
\newcommand{\vr}{{\mathrm{v}}}
\newcommand{\uu}{\bm{\mathrm{u}}}
\newcommand{\ur}{{\mathrm{u}}}


\newcommand{\abs}[1]{\left\vert #1 \right\vert}
\newcommand{\snk}{\sigma^2_{\bm{n},\bm{k}}}
\newcommand{\snj}{\sigma^2_{\bm{n},\bm{j}}}

\newcommand{\Rnj}{R_{\bm{n},\bm{j}}}
\newcommand{\Rnk}{R_{\bm{n},\bm{k}}}

\newcommand{\HH}{\mathrm{H}} % a set of notations for Holder spaces
\newcommand{\Ha}{\HH_{\alpha}}
\newcommand{\Hab}{\HH_{\alpha,\beta}}
\newcommand{\Habo}{\HH_{\alpha,\beta}^o}
\newcommand{\Hao}{\HH_{\alpha}^o}
\newcommand{\m}{\mathrm{m}}
\newcommand{\s}{\bm{s}}

\newcommand{\bb}{\bm{\beta}}
\newcommand{\bx}{\mathbf{x}}
\newcommand{\indf}[1]{\mathbf{1}\left( #1 \right)}

\title[CAPM]{CAPM ir akcijų portfelio konstravimas}
%\shorttitle{}
\author[ ]{Monika Šeštokaitė ir Simona Plonytė}
\institute[Vilnius University] {
    
    Vilniaus Universitetas
    \and
    
   Matematikos ir informatikos fakultetas
\and
Ekonometrinės analizės katedra
 }
\date{2011 gruodžio 8 d.}

\begin{document}
\begin{frame}
    \titlepage
\end{frame}

\begin{frame}
    \frametitle{CAPM} 
    \begin{itemize}
	\item Kapitalo aktyvų įvertinimo modelis (CAPM) įvertina aktyvo investicinės grąžos ir rizikos santykį tiriant vertybinių popierių rinkoje esančių akcijų pajamingumą.
    \item Kintamieji: $r_a$ -- a aktyvo pelningumas/grąža; 

$r_m$ -- rinkos akcijų pelningumas; 

$r_f$ -- nerizikingo aktyvo (iždo vekselio) grąža; 

$r_a-r_f$ -- a aktyvo rizikos premija; 

$r_m - r_f$ -- rinkos rizikos premija.
    \end{itemize}
\end{frame}
\begin{frame}
    \frametitle{Regresinis modelis} 
    \begin{itemize}
	\item $r_a - r_f = \alpha + \beta \times (r_m - r_f)$
	\item Koeficientai: $ \alpha $ parodo aktyvo vertę,\\ $ \beta $ -- aktyvo pelningumo priklausomybę nuo bendro rinkos pelningumo.
	

	
%\item $\beta = \frac{\sigma_{am}}{\sigma^2_m}$
%\item Čia $\sigma_{am}$ -- a aktyvo ir rinkos rizikos premijų kovariacija; 

%$\sigma^2_m$ -- rinkos rizikos premijos dispersija


    \end{itemize}
\end{frame}
\begin{frame}
    \frametitle{Duomenys} 
    \begin{itemize}
 	\item Aktyvai: Auksas, Microsoft Corp. akcijos, Apple Inc. akcijos, S\&P 500 indeksas, JAV 30-ies dienų iždo vekseliai;
	\item 2001 m. rugpjūčio 1 d. -- 2011 m. liepos 29 d.;
	\item Dieniniai duomenys.

    \end{itemize}
\end{frame}
\begin{frame}
    \begin{itemize}
    \frametitle{Aktyvų rizikos premijos priklausomybė nuo rinkos rizikos premijos} 
\item Modelis: $r_a - r_f = \alpha + \beta \times (r_m - r_f)$
	
\begin{table}[ht]
\begin{center}
    \begin{tabular}{ | l | l | l | }
    \hline
    Aktyvas & \beta & $ R^2 $ \\
    \hline
  Microsoft Corp. akcijos & 1.0039 & 0.5197 \\
	Apple Inc. akcijos  & 1.0270 & 0.3012 \\
  Auksas & −0.023 & 0.0007 \\
	\hline
    \end{tabular}
\end{center}
\end{table}
nzn ar grazu reikia!!!!!!!!!!!!!!!!!!!!!!!!!!!!!


	    \end{itemize}
\end{frame}
\begin{frame}
    \frametitle{Portfelis iš dviejų aktyvų}
    \begin{itemize}
	\item Formulė : a^{\ast}= \frac{\sigma_y^2 - r_{xy}\sigma_x \sigma_y}{\sigma^2_x + \sigma^2_y - 2r_{xy}\sigma_x \sigma_y} \label{form}


\begin{table}[ht]
\begin{center}

\resizebox{11cm}{!} {
    \begin{tabular}{ | l | l | l | l | }
    \hline
    Portfelio nr. & 1 & 2 & 3\\
    \hline
    I aktyvas; \% & Auksas; 81,17 \% & Microsoft Corp. akcijos; 74,92 \% & Auksas; 71,12 \% \\
    \hline
    II aktyvas; \% & Apple Inc. akcijos; 18,83 \% & Apple Inc. akcijos; 25,08 \% & Microsoft Corp. akcijos; 28,88 \% \\
    \hline
    Rizika & 0.0104 & 0.0180 & 0.0096 \\
    \hline
    Grąža & 0.00099 & 0.00060 & 0.00062 \\
    \hline
    \beta & 0.16243 & 1.0111 & 0.26268\\
	\hline	
    \end{tabular}
    }

\end{center}
\end{table}

\end{itemize}
\end{frame}
\begin{frame}
    \frametitle{Portfelis iš trijų aktyvų}
    \begin{itemize}
	\item 7.764768 \% Apple Inc. akcijų, 23.19523 \% Microsoft Corp. akcijų ir
69.04 \% aukso;
	\item Portfelio $\beta= 0.2967$, grąža -- 0.000733.
%	\item Note $0<\alpha<1/2$, since $W$ ``lives only'' in $\Hao([0,1])$ for
%	    $\alpha<1/2$.
    \end{itemize}
\end{frame}

\begin{frame}
    \frametitle{Išvados} 

    \begin{itemize}
\item Geriausia investicija  -- auksas;

\item Geriausi portfeliai: 
\begin{itemize}
	\item portfelis iš aukso ir Apple Inc. akcijų (geriausias esant nestabiliai rinkai);
	\item portfelis iš Microsoft Corp. ir Apple Inc. kompanijų akcijų (geriausias pakilimo laikotarpiu).
\end{itemize}
    \end{itemize}

\end{frame}


\begin{frame}
    \frametitle{Literatūra} 
    \begin{itemize}
	\item 
\item 
 \item 
   \item  
\item 
\item 
    \end{itemize}
 \end{frame}
 



\end{document}
